
\section{Fairness-for-Event-B}
\label{sec:Fairness-for-Event-B}

% K: fixme

\subsection{Introduction}
Fairness is a property of unbounded nondeterminism, which ensure the system fairly take the transactions when they are enabled, to avoid the situation that some transactions are always ignored. For example, a randomize number generator must be able to eventually generate every value within its range. Fairness can't be used on systems with finite executions.

\paragraph{Weak Fairness}
We say that an event i is weakly fairness (WF) iff it infinitely often disabled or infinitely often taken.
\begin{center}
WF(i)$\iff\Always\Eventually\neg{g}_{i}\lor\Always\Eventually{taken}_{i}$ 
\end{center}

\paragraph{Strong Fairness}
We say that an event i is strong fairness (SF) iff it eventually disabled forever or infinitely often taken.
\begin{center}
SF(i)$\iff\Eventually\Always\neg{g}_{i}\lor\Always\Eventually{taken}_{i}$ 
\end{center}
However, by working on the definitions for weak fairness and strong fairness, we can easily find that weak fairness imply strong fairness. Weak fairness is more detailed that strong fairness.  
\begin{center}
WF(i)$\Implies$SF(i) 
\end{center}

\paragraph{Event Merging}
For some situations we only have the fairness constraints for a set of events instead of fairness constraints for each of them. For example, Peter wants to travel all over the world, once he arrive a new place, he might stay there and spent several days for sightseeing. In this example, assume having an one-day sightseeing is event $E_{sightseeing}$, moving on air is event $E_{air}$ and moving by train is event $E_{train}$. By illustrating an event $E_{travel}$, as a merged event for moving to a new place by whatever mean. The following constraint
\begin{center}
WF($travel$) 
\end{center}
must be satisfied to ensure he can move from one place to the others rather than spending infinite days sightseeing somewhere without moving.\\
Since he does have any concern for choosing transportation, it's not necessary to ensure
\begin{center}
WF($air$) or WF($train$) 
\end{center}
which force him always eventually choose airplane or train.\\
For this type of constraints, we introduce following rule for merging event:\\
For event $E_{i}$ and event $E_{j}$, where i,j $\in$ I, the merged event $E_{i\lor{j}}$ satisfy:
\begin{center}
$
\begin{array}{lcl}
g_{i\lor{j}} & = & g_{i}\lor{g_{j}}\\
a_{i\lor{j}} & | & a'_{i\lor{j}} \land \\
             &   & (((g_{i}\land\neg{g_{j}})\Implies (a'_{i\lor{j}} = a_{i}))\\
             &   & \lor ((\neg{g_{i}}\land{g_{j}})\Implies (a'_{i\lor{j}} = a_{j}))\\
             &   & \lor (({g_{i}}\land{g_{j}})\Implies (a'_{i\lor{j}} = a_{j}\lor{a_{j}})))
\end{array}
$
\end{center}
The merged event $E_{j}$ have a disjoint guard of event $E_{i}$ and event $E_{j}$, and $a_{i\lor{j}}$ is the corresponding available action from of event $E_{i}$ and event $E_{j}$, if multiple guards satisfied, the action will be the disjoint of $a_{j}$ and ${a_{j}}$. \\
Some properties for fairness constraints of merged events are:\\
Assume i,j $\in$ I, event $E_{i\lor{j}}$ is the merged event from $E_{i}$ and $E_{j}$
\begin{center} 
WF(i) $\lor$ WF(j) $\Implies$  WF($i\lor{j}$)\\
SF(i) $\lor$ SF(j) $\Implies$  SF($i\lor{j}$)
\end{center}
\paragraph{Event Splitting}
In an opposite way, we can also split an event into several mutually exclusive sub-events, by refining them with different mutually exclusive strengthen guards which fullfill the original guard. \\
For event h where h $\in$ I, event i and event j are splitted events of h iff:
\begin{center} 
$g_{i} \lor{g_{j}} = g_{h}$\\
$g_{i}\land{g_{j}} = \varnothing$\\
$a_{i} = a_{j} = a_{h}$
\end{center}
Note that the fairness constraints the original event hold can't be inherited by any one of its sub-events, for the same reason explained in event merging part. However, similar to the outcome of the event merging process, we can say a event is fair if we can ensure that one of its sub-events is fair.
\begin{center} 
WF(i) $\lor$ WF(j) $\Implies$  WF(h)\\
SF(i) $\lor$ SF(j) $\Implies$  SF(h)
\end{center}
Where event i and j are sub-events of event h.\\\\
Currently Event-B does not have suppport for LTL, most of properties described by LTL (including fairness) can't be assumed or proved by Event-B. In the rest parts of this report, we will extend the original Event-B method and show the usage of fairness assumptions in proving response, a basic property of unbounded nondeterminism.\\


 	$\sideset{_1^2}{_3^4}\prod_a^b$
 	$\bigvee_{i=_1}^n E_i$